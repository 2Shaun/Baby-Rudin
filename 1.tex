\documentclass[12pt]{article}

\usepackage{bbding}
\usepackage{amsthm}
\usepackage{amssymb}
\usepackage{amsmath}
\usepackage{amsfonts}

\author{Tommy O'Shaughnnesy}
\title{Principles of Mathematical Analysis Notes}
% Optional commands: \email and \bottomtext
%\email{thomas_oshaughnessy@mymail.rcbc.edu}
%\bottomtext{}

\begin{document}
\maketitle
\section{Chapter 1 Exercises}
\subsection{}
\begin{proof}
				To prove (a) by contradiction, let $r + x = \frac{p}{q}$ for some $p,q \in \mathbb{Z}$.
				It follows that $x = \frac{p-rq}{q}$ which contradicts $x \in \mathbb{I}$.
				Thus, $r + x \in \mathbb{I}$. \\
				Similarly, to prove (b) by contradiction,
				let $rx = \frac{p}{q}$. It follows that $x = \frac{p}{qr}$ 
				which contradicts $x \in \mathbb{I}$. Thus, $rx \notin \mathbb{Q}$.
\end{proof}
\subsection{}
\begin{proof}
				To prove this by contradiction, assume $\frac{p^{2}}{q^{2}}=12$. It follows that 
				\begin{gather*}
								p^{2}=12q^{2}=2^{2}\cdot3^{1}\cdot q^{2}.
				\end{gather*}
				
								By the fundamental theorem of arithmetic, $p^{2}$ must factor into a
				product of primes of even multiplicity. By the same argument, $q^{2}$ must factor into
				a product of an even multiplicity of $3$, contradicting the unique factorization of $p$.
				Therefore, the assumption is false and $\sqrt{12}\in\mathbb{I}$.
\end{proof}
\subsection{}
\begin{proof}
				To prove (a), by (M5)
				\begin{gather*}
					\frac{1}{x}xy=\frac{1}{x}xz=y=z.
				\end{gather*}
\end{proof}
\begin{proof}
				To prove (b), by (M4)
				\begin{gather*}
								x1=x=xy.
				\end{gather*}
				By (a), $1=y$.
\end{proof}
\begin{proof}
				To prove (c), by (M5)
				\begin{gather*}	
								x\frac{1}{x}=1=xy.
				\end{gather*}	
				By (a), $y=\frac{1}{x}$.
\end{proof}
\begin{proof}
				To prove (d), by (M5)
				\begin{gather*}
								\frac{1}{x}1/(1/x)=1=\frac{1}{x}x.
				\end{gather*}
				By (a), $1/(1/x)=x$.
\end{proof}
\subsection{}
\begin{proof}
				By the definition of lower bound, $\alpha\leq x$ for every $x\in E$.
				By the definition of upper bound, $x\leq\beta$. Combining inequalities, 
				$\alpha\leq x\leq\beta$, which implies $\alpha\leq\beta$.
\end{proof}
\subsection{}
\begin{proof}
				Since $A$ is bounded below, let $\alpha=\inf A$. By the definition of greatest lower bound,
				$x\geq\alpha$ for all $x\in A$. It follows that $-x\leq-\alpha$ for all $-x\in A$.
				Let $-x=y$ for some $y\in-A$. Therefore, $y\leq-\inf A=\sup -A$, for all $y\in-A$.
				It follows that
				\begin{gather*}
							(-1)-\inf A=(-1)\sup -A=\inf A=-\sup-A.
				\end{gather*}
\end{proof}
\subsection{}
\begin{proof}
				To prove (a), first notice that $m=rn$. By Corollary 1.21, it follows that
				\begin{gather*}
								(b^{m})^{\frac{1}{n}}=(b^{rn})^{\frac{1}{n}}=b^{r}.
				\end{gather*}
				Theorem 1.21 shows that $b^{r}=(b^{m})^{\frac{1}{n}}$.
\end{proof}
\begin{proof}
				To prove (b) by Corollary 1.21, it follows that
				\begin{gather*}
								(b^rb^s)^\frac{1}{r}=b\cdot b^{\frac{s}{r}}=b^{\frac{s}{r}+1}=b^{\frac{s+r}{r}}=
								(b^{s+r})^{\frac{1}{r}}.
				\end{gather*}
				Thus, by Theorem 1.21, $b^rb^s=b^{r+s}$.
\end{proof}
\begin{proof}
				To prove (c), first we will prove that $B(x)$ has the least upper bound property. Let 
				$\varepsilon>0$, $b=1+\varepsilon$, and $t=x$. It follows that $b^t\leq b^x$, therefore
				$b^t\in B(x)$ and $B(x)$ is not empty. Next, notice that $x<x+\varepsilon$. Since $b>1$,
				\begin{gather*}
								b^{x-\varepsilon}<b^t\leq b^x<b^{x+\varepsilon}. 
				\end{gather*}
				Notice that 
				$b^{x+\varepsilon}\notin B(x)$ and $b^{x-\varepsilon}$ is not an upper bound, because 
				$b^{x+\varepsilon}\nleq b^x$ and $b^{x-\varepsilon}<b^x\in B(x)$. By Definition 1.8,
				$b^x = \sup B(x)$ for any $x\in\mathbb{R}$.
\end{proof}
\begin{proof}
				To prove (d), ...
\end{proof}
\subsection{}
\begin{proof}
				To prove (a) using the equality
				\begin{gather*}
								b^n-a^n=(b-a)(b^{n-1}+b^{n-2}a+...+a^{n-1})
				\end{gather*}
				where $a=1$, yields
				\begin{gather*}
								b^n-1=(b-1)(b^{n-1}+b^{n-2}+...+1).
				\end{gather*}
				Notice that there are $n$ terms in right side of the equality. Since $b>1$, 
				it follows by (D) that
				\begin{gather*}
								b^n-1=b^{n-1}(b-1)+b^{n-2}(b-1)+...+(b-1)\geq n(b-1).
				\end{gather*}
\end{proof}

\section{Addition in the Real Number Field}
\begin{proof}
Let $\alpha$ and $\beta$ be cuts, such that $\alpha\subset\beta$. 
Let $r\in\alpha$ and $s\in\beta$. The cut defined by $\alpha+\beta$ is thus the set of all $r+s$. 
Since $\alpha\in\beta$, by (II), $r-s\in\alpha$. Since $r=r-s+s$, we can say $(r-s)+(s)
\in\alpha+\beta$ and therefore $r\in\alpha+\beta$.\\
\end{proof}

\begin{proof}
To verify that $\alpha + \beta$ satisfies (II), for some $r' \in \alpha$ such that $r < r'$ and 
$s' \in \beta$ such that $s < s'$. It follows that $r + s < r' + s' \in \alpha + \beta$.
\end{proof}

\end{document}
