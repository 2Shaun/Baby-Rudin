\documentclass[12pt]{article}

\usepackage{bbding}
\usepackage{amsthm}
\usepackage{amsfonts}

\author{Tommy O'Shaughnnesy}
\title{Principles of Mathematical Analysis Notes}
% Optional commands: \email and \bottomtext
%\email{thomas_oshaughnessy@mymail.rcbc.edu}
%\bottomtext{}

\begin{document}
\maketitle
\section{Addition in the Real Number Field}
\begin{proof}
Let $\alpha$ and $\beta$ be cuts, such that $\alpha\subset\beta$. 
Let $r\in\alpha$ and $s\in\beta$. The cut defined by $\alpha+\beta$ is thus the set of all $r+s$. 
Since $\alpha\in\beta$, by (II), $r-s\in\alpha$. Since $r=r-s+s$, we can say $(r-s)+(s)
\in\alpha+\beta$ and therefore $r\in\alpha+\beta$.\\
\end{proof}

\begin{proof}
To verify that $\alpha + \beta$ satisfies (II), for some $r' \in \alpha$ such that $r < r'$ and 
$s' \in \beta$ such that $s < s'$. It follows that $r + s < r' + s' \in \alpha + \beta$.
\end{proof}
\section{Chapter 1 Exercises}
\subsection{1.1}
\begin{proof}
				To prove (a) by contradiction, let $r + x = \frac{p}{q}$ for some $p,q \in \mathbb{Z}$.
				It follows that $x = \frac{p-rq}{q}$ which contradicts $x \in \mathbb{I}$.
				Thus, $r + x \in \mathbb{I}$. \\
				Similarly, to prove (b) by contradiction,
				let $rx = \frac{p}{q}$. It follows that $x = \frac{p}{qr}$ 
				which contradicts $x \in \mathbb{I}$. Thus, $rx \notin \mathbb{Q}$.
\end{proof}
\subsection{1.2}


\end{document}
